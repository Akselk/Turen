\chapter{Asia}

sett inn bilde


Da er det endelig duket for siste verset i min reisesaga. Fikk med meg
sosio-log/nom? (sånn dærre sosialarbeidår) og musiker Simon Timenes.  

(synkronutbrud sopp)
\section{One Night in Bangkok}

Vi gikk ut av flyet og kjente den orientalske midtdagssolen steke over
den vinterbleke huden vår. Luftfuktigheten var som en klam vegg som
omfavnet hele kroppen\ldots Neida. Fra spøk til revolver. Det var
jævlig varmt.\\

Bangkok er kaos, god gatemat, hvite jenter i elefantbukser (Vi
kommer tilbake til dette) og same same but different. Tuk tuken tar
deg der han føler du burde dra. Men sånne småting er ikke verd å melke
en historie ut av. Det som må nevnes er Songkran

bilde (fastfood)

\subsection{Om Songkran}

Songkran, eller vannfestivalen om du vil, begynte som en hellig rite.
I riten inngikk det at vann skulle drysses på veien før monkene
gikk der. Noe i den duren i det minste. Det var iallefall (fra nå av
jaffal) veldig høytidelig og
hadde sikkert en form for symbolsk verdi og mening. Det meste
forandrer seg med tiden. Og i Songkran sitt tilfelle har det forbedret seg. Det har nemlig eskalert til en ``city-wide''
vannkrig med epicenter (uttales epic center) i Kaoh san road. Simon og jeg
bevæpnet oss med to stk. super soacker 1500, to bøtter energi og
entret striden! Vått ble det.\\

(bilde med supersoaker)

\subsection{Point of no return}

Etter dette vann- og alkoholkalaset fikk vi akkurat hevet av oss deg
våte tøyet før det bar til en 18 timers transit til Japan. Ikke før
rett etter bagasjen var sjekket inn oppdaget jeg at skoene mine enda
var klissvåte\ldots De fleste har et forhold til: ``the
point of no return''. Spesielt menn som har vært på et par gallaer
i sitt liv.\\ På galla går gjerne damer i nette kjoler. Og alle og
enhver med en 5- lem kan jo foretrekke dette. Problemet er at
temperaturen i rommet ofte blir skrudd opp så ikke damene skal fryse i
de nette kjolene. For menn er det to plagg som gjelder i enhver
finere sammenheng. Det er dress og det er bunad. Ingen av dem er
særlig nette. Så det er ikke lenge man må være i det
ekvatortempererte gallalokalet før man kan finne på både Frodo, Sam og
resten av brorskapet under armene. For å forhindre dette må man ta av
jakka lynrask etter man kommer inn i lokalet. For når de først er
kommet er vi på ``the-point-of-no-return'' da må man bare pine seg
gjennom kvelden med dressjakka på. Dette er forøvrig også en av
grunnen til at damer er mer glad enn menn i å danse på galla. \\

Tilbake til skoene mine. Jeg er sjekket inn på flyplassen. Alt jeg har
er en liten sekk med pc. Ingen skift. The point of no return var i går
kveld. Det er ikke mulig å ta dem av. Folk sitter for nærme. Jeg skal
ikke kjede dere med detaljene, men avslutter med å si at etter 18
timer uten å ta dem av kunne vi gjort baneheia om til en regnskog ved
å legge skoene der en natt.



\section{Japan}
Klokken 10:00 hadde vi landet, hentet bagasjen, tatt buss til sentrum
og sjekket inn på hostellet. Nå var det på tide med frokost. De første
10 minuttene på matjakt lærte vi to ting om Japan.

\begin{enumerate}
	\item Folk snakker ikke engelsk
	\item Butikker er fysisk ``åpne'' før de er åpne i praksis.
\end{enumerate}

Etter å hadde blitt kastet ut to steder med japanske
gloser trådde vi litt forsiktig inn i det tredje. Vi åpnet
døren og startet med et forsiktig
\begin{dialouge}
	\item ``Hallo\ldots?''
%	\item
	\item ``IRRASHAIMASE!'' 
\end{dialouge}
\\Ble skreket tilbake av 10 japanere i	unison.	
Vi var akkurat på vei til
å pile ut med hallen mellom bena da ei vertinne kom å smilte pent og
viste oss til et bord. Vi satt oss ned ganske fortumlet og tenkte,
``jaja velkommen til Japan si'. Ironisk nok Noe lokalet nettopp hadde ønsket oss i
unison. 

\subsection{Om japanere}
Dette er den mest ambisiøse ``om'' til nå. Skal imidlertid prøve å
ramse noen litt særegne trekk med japanerne. Legg til alle disse
trekkene når det senere blir fortalt om stedene vi besøkte. Først da
skjønner du hvilke spesiell opplevelse dette landet er.\\


\subsubsection{1.}
For å da det første først. Japanere blir ikke åpenlyst
sinte. Å vise ydmykhet er den japanske vei. Å som alle
japanske veier gjør de dette til det ekstreme. Å være
ydmyk til det arrogante er en kunst ikke alle folkeslag kan
mestre. 
Da vi ble kastet ut av butikkene vi først kom til
Tokyo ble vi ikke kastet ut med banneord og sinte
gloser som i Colombia. Neida. Damen unnskyldte seg
om og om igjen for at butikken ikke var åpen og bukket
oss ut døra. Å bukke er ydmykt og i Japan er det da
høflig. Det er selvfølgelig mest høflig å bukke
sist. Noe som gir noen artige men slitsomme
bukke-mens-man-går-og-snur-seg seanser. 

\subsubsection{2.}
\textbf{They way off}\ldots.
Alt som lages i Japan er i verdensklasse. For når
japanere først gjør noe kan du garantere deg på at de
legger hele sin ære i det. Skulle sagt sjela, men ære
`er langt viktigere for en japaner. `The way of
sushi\ldots'
``the way og knives\ldots'' ``The way og beef\ldots'' En dansk
knivselger som har bodd i Japan 20 år i Japan
mente at the way ikke var krafig nok for knivmakerne, her
var det snakk om ``the life off''. En hel landsby går
inn i knivmakingen. Hver del av landsbyen har
ansvar for hver sin del av prosessen og rollene går i
arv. I dokumentaren
Jiro dreams of sushi kan du se et fint eksempel på
dedikasjon.. 
Spesielt interesserte kan se hele. Humoren ligger
imidlertid i at sønnen til Jiro har vært lærling i 40
år og får bare koke ris. 

\subsubsection{3.}
Kredittkortet ditt er
en tillitserklæring og det skal man ikke ta lett på.
Alle gjenstander forveksles med to henter, øyekontakt og et
lite bukk. 

\subsubsection{4.}
Japanere står mye i kø. Til den grensen man skulle
tro det var en hobby. Og ikke en norsk
kø-i-klynge. Enten står du rett bak personen fremfor deg
, eller så er du ikke i køen. Dette gjelder
overalt: på butikken, på undergrunnen, barn som venter på skolebussen,
og på rødt lys. Om det er 2 pannekakeboder er
det 30 min kø til den ene og den andre står tom. Tror
det er på tankegangen: ``om andre står der må det være verdt det.''  


\subsubsection{5.}
Yrket ditt har alltid en uniform. Og ikke bare en pique eller skjorte
som er vanlig i Norge. Her er taxisjåføren fullt utstyrt i dress, hatt
og hvite hansker. Bilen er forresten også skinnende ren. De vaskes
hver dag.

\subsubsection{6.}
Regler skal følges. Og ikke bare for regelens hensikt. En regel er en
hensikt i seg selv. Er det rødt lys går \underline{ingen} over veien.
Uavhengig om at det er null biler på denne tiden av døgnet.

\subsubsection{7.}
Da jeg nevnte til resepsjonisten at vi
ikke bruker kontanter i Norge så hun rart på meg og spurte: ``Hvordan
betaler dere på automatene?'' \underline{Alt} kan kjøpes i automater! 

\subsection{8.}
Inne bruker man tøfler. Og Gud forby at du bruker samme tøfler i stua
som på doen. Står klare egne dotøffler til det gjøremålet!

\subsection{Tokyo}

\subsubsection{Buried alive}
For å spare penger sjekket vi inn på et kapselhostell. Et
kapsellhostell består av et rom med mange kammer eller ``kapsler'' inn
i veggen som man sover i. Alle av den eldre generasjonen syntes dette
høres helt forferdelig ut. Enten er de litt for godt vant, eller så
har de ikke lyst å fremskynde prosessen med å legge seg i en kiste!
Helt ærlig er kapselene den beste formen for dorm. Man har privatliv,
det er rent, og de fleste hostellene er nye og har brukt mye penger på dusj og
fellesareal. 


\subsubsection{Mote-Mekka}
Tokyo er visstnok verdenledende på mote (the way off\ldots). Dette
hadde noen entusastiske Oslo-kompiser nevn til Simon mer enn èn gang.
Tokyoanere kjører sin helt egen stil. Og de mest dedikerte tokyoanerne
finner man i Harajuku. Her fikk vi kjøpt et par snasne hatter for å
blende inn i mengden. Har bestemt meg for at bare egne bilder skal inn i
denne boka så gjør et google-søk på Harajuku fashion og se hva det går
i!

Taxtadori 
