\chapter{Svalbard}\\ {\footnotesize \textit{08--20 mars}}

\begin{figure}[H]
	\centering
	\vspace{-2cm}
\noindent\makebox[\textwidth]{\includegraphics[width=\paperwidth]{svalbardvenstre}}
	\caption*{\textbf{Om du tror Svalbard minner om det norske høyfjellet tar du feil.
Svalbard gikk aldri ut av istiden.
Isen
her er gammel. Veldig gammel. I flere tusen år har isen vært her. }} 

	\label{fig:stianglacier}
\end{figure}


\begin{figure}[H]
	\centering
	\vspace{-2cm}
\noindent\makebox[\textwidth]{\includegraphics[width=\paperwidth]{svalbardhoyre}}
	\caption*{\textbf{Den samme isen som
slepet det norske grunnfjellet til fjorder og daler. Her smeltet den
aldri. Den kjemper en evig kamp mot det arktiske havet som uavbrutt slår inn i alle ender. Å se denne isfronten på nært hold kan
verken beskrives med mine ord eller bilder.} Tunaglacier av Stian
Aadland}
\label{fig:venstre}
\end{figure}
