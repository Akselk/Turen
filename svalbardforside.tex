\chapter*{Svalbard}\\ {\footnotesize \textit{08--20 mars}}

\begin{figure}[H]
	\centering
%	\includegraphics{tunaglacierstian}
\noindent\makebox[\textwidth]{\includegraphics[width=\paperwidth]{tunaglacierstian}}
	\caption*{\textbf{Om du tror Svalbard minner om det norske høyfjellet tar du feil.
Svalbard gikk aldri ut av istiden.
Isen
her er gammel. Veldig gammel. I 20000 år har isen vært her.  
Den samme isen som
slepet det norske grunnfjellet til fjorder og daler. Her smeltet den
aldri. Den kjemper en evig kamp mot det arktiske havet som uavbrutt slår inn i alle ender. Å se denne isfronten på nært hold kan
verken beskrives med mine ord eller bilder.} Tunaglacier av Stian
Aadland}

	\label{fig:stianglacier}
\end{figure}


%i\begin{figure}[H]
%	\centering
%\noindent\makebox[\textwidth]{\includegraphics[width=6.125in]{tunaglaciervenstre}}
%	\caption*{}
%\label{fig:venstre}
%\end{figure}
%\textbf{Om du tror Svalbard minner om det norske høyfjellet tar du feil.
%Svalbard gikk aldri ut av istiden.
%Isen
%her er gammel.  Veldig gammel. I 20000 år har isen vært her.} 

%\chapter*{\\ \\}\\ {\footnotesize \textit{\\ \\ \\ \\ \\ \\ \\ \\ \\ \\ \\ \\ }}
%\begin{figure}[H]
%	\centering
%\noindent\makebox[\textwidth]{\includegraphics[width=6.125in]{tunaglacierhoyre}}
%	\caption*{}
%\label{fig:venstre}
%\end{figure}
%\textbf{Den samme isen som
%slepet det norske grunnfjellet til fjorder og daler. Her smeltet den
%aldri. Den kjemper en evig kamp mot det arktiske havet som uavbrutt slår inn i alle ender. Å se denne isfronten på nært hold kan
%verken beskrives med mine ord eller bilder.}


